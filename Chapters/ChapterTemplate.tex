% Chapter Template


\chapter{Chapter Title Here} % Main chapter title
%\chapter[toc version]{doc version}
%\chaptermark{version for header} %Short version of the title for the headeer
\label{ChapterX} % Change X to a consecutive number; for referencing this chapter elsewhere, use \ref{ChapterX}

% Write text in here
% Use \subsection and \subsubsection to organize text

Welcome to the tutorial on how to use this thesis model. This is not to teach you how to use \LaTeX. For that read a tutorial. But this aims to teach you how to do the basic stuff you will need in order to produce a decent document.

\section{Citations}
You can add extra info to you references, like \cite[chapter 3]{Nobody06}

\section{Figures}

Let us start with a figure with two subfigures like in \ref{fig:FCUPfatCat}. 

\begin{figure}
	\centering
	\begin{subfigure}{.49\textwidth}
  		\centering
  		\includegraphics[width=.95\linewidth]{Figures/ChapterTemplate/20160517_123603.jpg}
  		\caption{FCUP's fat cat doing what cats do.}
	\end{subfigure}%
	\hfill
	\begin{subfigure}{.49\textwidth}
  		\centering
 		 \includegraphics[width=.95\linewidth]{Figures/ChapterTemplate/20160517_123609.jpg}
 		 \caption{FCUP's fat cat resting.}
	\end{subfigure}
	\caption{\label{fig:FCUPfatCat}FCUP's fat cat.} 
\end{figure}


Or two figures side by side like \ref{fig:FCUPfatcatSide1} and \ref{fig:FCUPfatcatSide2}.

\begin{figure}
\centering
\begin{minipage}{.49\textwidth}
  \centering
  \includegraphics[width=.95\linewidth]{Figures/ChapterTemplate/20160517_123603.jpg}
  \captionof{figure}{\label{fig:FCUPfatcatSide1}FCUP's fat cat doing what cats do.}
\end{minipage}%
\hfill
\begin{minipage}{.49\textwidth}
  \centering
  \includegraphics[width=.95\linewidth]{Figures/ChapterTemplate/20160517_123609.jpg}
  \captionof{figure}{\label{fig:FCUPfatcatSide2}FCUP's fat cat.} 
\end{minipage}
\end{figure}

\section{Math}


The following equation uses a custom mathematical operator defined in line 166 of the stock main.tex:
\begin{equation}
\begin{aligned}
			\meshgrid_{\mathbf{x}_{1},\mathbf{x}_{2}}\mathbf{x}_{1}&=\begin{bmatrix}a_{1} & b_{1} & c_{1}\\
a_{1} & b_{1} & c_{1}
\end{bmatrix}\\
			\meshgrid_{\mathbf{x}_{1},\mathbf{x}_{2}}\mathbf{x}_{2}&=\begin{bmatrix}a_{2} & a_{2} & a_{2}\\
b_{2} & b_{2} & b_{2}
\end{bmatrix}
\end{aligned}
\end{equation}

And this is an equation with multiple lines:
\begin{equation}
\begin{aligned}
&I_{0}=I^{\prime}+I^{\prime\prime}\cos(\varPsi)   \\
&I_{\pi/2}=-I^{\prime\prime}\sin(\varPsi)                \\
&I_{\pi}=I^{\prime}-I^{\prime\prime}\cos(\varPsi)   \\
&I_{3\pi/2}=I^{\prime\prime}\sin(\varPsi)
\end{aligned}
\end{equation}

